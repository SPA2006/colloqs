\section{Понятие числового ряда. Критерий Коши. Необходимое и достаточное условие сходимости рядов с неотрицательными членами}

Рассмотрим числовую последовательность $\{u_k\}_{k=1}^\infty$, формально просуммируем все её члены

\begin{equation}
	u_1 + u_2 + ... + u_k \equiv \displaystyle\sum_{k = 1}^\infty u_k
\end{equation}

полученное выражение назовём числовым рядом, в котором $u_k$ - общий член ряда, знак $\Sigma$ означает сумму; сумму первых $n$ слагаемых, $S_n = \displaystyle_{k = 1}^n u_k$ назовём $n$-й суммой ряда (1).

\begin{definition}
	Ряд (1) называется сходящимся, если сходится последовательность $\{S_n\}$ частичных сумм этого ряда.
\end{definition}

Предел $S$ последовательности $\{S_n\}$ называется суммой этого ряда. Для сходящегося ряда можно формально записать

\begin{equation*}
	S = \displaystyle\sum_{k  = 1}^\infty u_k
\end{equation*}

выражение

\begin{equation*}
	S - S_n = \displaystyle\sum_{k = n + 1}^\infty u_k \equiv r_n
\end{equation*}

называется $n$-м остатком ряда. Из определения предела последовательности получаем, что

\begin{equation*}
	\exists S: \forall\varepsilon > 0\: \exists N > 0: \forall n \geqslant N\quad |S_n - S| = |r_n| < \varepsilon
\end{equation*}

т.е. $r_n = o(1)$, остаток сходящегося ряда является бесконечно малой величиной.

Если конечного предела последовательности $\{S_n\}$ не существует, то ряд называется расходящимся. Формально это можно записать как

\begin{equation*}
	\forall S \: \exists\varepsilon > 0: \forall N \: \exists n \geqslant N: |S_n - S| = |r_n| \geqslant \varepsilon
\end{equation*}

\textbf{Критерий Коши}

\begin{theorem}
	(\textbf{критерий Коши}) Для того чтобы последовательность $\{S_n\}$ была сходящейся, необходимо и достаточно, чтобы
	
	\begin{equation*}
		\forall\varepsilon > 0 \: \exists N: \: \forall n \geqslant N \: \forall p \in \mathbb{N} \quad |S_{n + p} - S_n| < \varepsilon
	\end{equation*}
	
	Если $\{S_n\}$ - последовательность частичных сумм ряда (1), то
	
	\begin{equation*}
		S_{n + p} - S_n = \displaystyle\sum_{k = n + 1}^{n + p} u_k
	\end{equation*}
	
	поэтому следующее утверждение получаем как следствие из предыдущего утверждения
\end{theorem}

\begin{theorem}
	(\textbf{критерий Коши для ряда}) Для того чтобы ряд (1) сходился, необходимо и достаточно, чтобы
	
	\begin{equation}
		\forall\varepsilon > 0 \ \exists N: \forall n \geqslant N \ \forall p \in\mathbb{N} \quad \left|\displaystyle\sum_{k = n + 1}^{n + p} u_k \right| < \varepsilon
	\end{equation}
\end{theorem}

\begin{follows}
	(\textbf{необходимое условие сходимости ряда}) Для сходимости ряда (1) необходимо, чтобы последовательность $\{u_k\}$ членов этого ряда была бесконечно малой, т.е. $u_k = o(1),\ k\rightarrow\infty$.
\end{follows}

\textbf{Доказательство}

Для доказательства этого утверждения достаточно заметить, что согласно теореме 2

\begin{equation*}
	\forall\varepsilon > 0 \ \exists N_0: \forall n \geqslant N_0 \ \forall p \in \mathbb{N} \quad \left| \displaystyle\sum_{k = n + 1}^{n + p} u_k \right| < \varepsilon
\end{equation*}

При $p = 1$ соотношение принимает вид $|u_{n + 1}|$ < $\varepsilon$. Положив $N = N_0 + 1$, получим, что

\begin{equation*}
	\forall\varepsilon > 0 \ \exists N\in\mathbb{N}: \forall n \geqslant N \quad |u_n| < \varepsilon
\end{equation*}

т.е. $u_k \rightarrow 0, k \rightarrow\infty$, что и требовалось доказать.

\textbf{Примеры рядов, для которых не выполняется необходимое условие сходимости ряда}

\begin{equation*}
	\displaystyle\sum_{k = 1}^\infty (-1)^k, \ \displaystyle\sum_{k = 1}^\infty (-1)^kk, \ \displaystyle\sum_{k = 1}^\infty \frac{2k^3}{3k^3 + k^2}
\end{equation*}

Таким образом, каждый из этих рядов расходится.

Стремление к нулю общего члена ряда является лишь необходимым, но не достаточным условием сходимости ряда. Например, рассмотрим гармонический ряд

\begin{equation*}
	1 + \frac{1}{2} + \frac{1}{3} + ... = \displaystyle\sum_{k = 1}^\infty \frac{1}{k}
\end{equation*}

Докажем, что гармонический ряд - расходящийся. Воспользуемся критерием Коши

\begin{equation*}
	\exists\varepsilon > 0: \quad \forall N\in\mathbb{N} \ \exists n \geqslant N, \ \exists p \in\mathbb{N}: \quad \left| \displaystyle\sum_{k = n + 1}^{n + p} u_k \right| \geqslant \varepsilon
\end{equation*}

В последней сумме $p$ слагаемых. Для всех $k \leqslant n + p$ выполнено неравенство $\frac{1}{k} \geqslant \frac{1}{n + p}$, положим $p = n$

\begin{equation*}
	\displaystyle\sum_{k = n + 1}^{n + p} \frac{1}{k} \geqslant \frac{1}{n + p} \cdot p \Bigl|_{p = n} = \frac{n}{2n} = \frac{1}{2}
\end{equation*}

Итак,

\begin{equation*}
	\forall\varepsilon \in \Bigl( 0, \frac{1}{2} \Bigr],\ \forall N \in \mathbb{N} \ \exists n\geqslant N, \ \exists p = n: \quad \left| \displaystyle\sum_{k = n + 1}^{n + p} u_k \right| \geqslant \varepsilon
\end{equation*}

То есть, согласно теореме 2, гармонический ряд расходится. Заметим, что в этом случае в качестве $n$ можно взять любое натуральное число, большее или равное $N$, а в качестве $\varepsilon$ - любое число из полуинтервала $(0, \frac{1}{2}]$.

\begin{note}
	Отбрасывание или добавление конечного числа членов ряда не влияет на сходимость или расходимость ряда, это следует из критерия Коши, так как для больших $n$ разность $S_{n + p} - S_n$ не изменится.
\end{note}

\begin{note}
	Если постоянная $c$ отлична от нуля, то ряды
	
	\begin{equation*}
		\displaystyle\sum_{k = 1}^\infty (cu_k) \ \mbox{и} \ \displaystyle\sum_{k = 1}^\infty u_k
	\end{equation*}
	
	сходятся или расходятся одновременно, так как последовательность $\{cS_n\}$ сходится тогда и только тогда, когда сходится последовательность $\{S_n\}$, это также можно вывести из критерия Коши.
\end{note}