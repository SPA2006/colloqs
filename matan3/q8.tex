\section{Признаки сходимости произвольных рядов (пр. Абеля)}

В интегральном исчислении важную роль играет формула интегрирования по частям. Приведём аналог этой формулы для рядов.

\begin{statm}
	Пусть $\{u_k\}$ и $\{v_k\}$ - две произвольные последовательности, $S_n = u_1 + u_2 + ... + u_n, S_0 = u_0$, $n$ и $p$ - два произвольных номера. Тогда справедливо тождество
	
	\begin{equation}
		\displaystyle\sum_{k = n}^{n + p} u_kv_k = \displaystyle\sum_{k = n}^{n + p - 1} S_k (v_k - v_{k + 1}) + S_{n + p}v_{n + p} - S_{n - 1}v_n
	\end{equation}
	
	называется тождеством Абеля или преобразованием Абеля.
	
	Если переписать тождество (32) в виде
	
	\begin{equation*}
		\displaystyle\sum_{k = n}^{n + p} v_k\Delta S_k = S_{n + p}v_{n + p} - S_{n - 1}v_n - \displaystyle\sum_{k = n}^{n + p -1} S_k\Delta v_k
	\end{equation*}
	
	где $\Delta S_k = S_k - S_{k - 1} = u_k, \Delta v_k = v_{k + 1} - v_k$, то достаточно наглядно просматривается связь этого тождества и формулы интегрирования по частям для определённых интегралов.
\end{statm}
\textbf{Доказательство}

Подставим в левую часть формулы (32)  вместо $u_k$ разность $u_k = S_k - S_{k - 1}$ и во второй из получаемых при этом заменим индекс суммирования $k$ и $k + 1$. Получим

\begin{equation*}
	\begin{gathered}
		\displaystyle\sum_{k = n}^{n + p}u_kv_k = \displaystyle\sum_{k = n}^{n + p}S_kv_k - \displaystyle\sum_{k = n}^{n + p} S_{k -1}v_k = \displaystyle\sum_{k = n}^{n + p}S_kv_k - \displaystyle\sum_{k = n - 1}^{n + p - 1} S_k v_{k + 1} = \\
		= S_{n + p}v_{n + p} + \displaystyle\sum_{k = n}^{n + p - 1}S_kv_k - \displaystyle\sum_{k = n}^{n + p - 1} S_kv_{k + 1} - S_{n - 1}v_n = \\
		= \displaystyle\sum_{k = n}^{n + p - 1} S_k(v_k - v_{k + 1}) + S_{n + p}v_{n + p} - S_{n - 1}v_n
	\end{gathered}
\end{equation*}

Откуда следует формула (32).

\begin{flushright}
	$\blacksquare$
\end{flushright}

Для применения критерия Коши сходимости числовых рядов тождество (32) удобно записать в следующем виде

\begin{equation}
	\displaystyle\sum_{k = n + 1}^{n + p} u_kv_k = \displaystyle\sum_{k = n + 1}^{n + p - 1} S_k(v_k - v_{k + 1}) + S_{n + p}v_{n + p} - S_nv_{n + 1},\quad p \neq 1
\end{equation}

А в случае $p = 1$ в правой части (33) первую сумму следует заменить нулём.

\textbf{Признаки сходимости}

Рассмотрим числовой ряд

\begin{equation}
	\displaystyle\sum_{k = 1}^\infty u_kv_k
\end{equation}

обозначим через $\{S_n\}$ последовательность частичных сумм ряда

\begin{equation}
	\displaystyle\sum_{k  = 1}^\infty u_k
\end{equation}

\begin{theorem}
	(\textbf{два признака Абеля}) \begin{enumerate}
		\item Пусть выполняются следующие условия
		
		\begin{enumerate}
			\item последовательность $\{S_n\}$ является ограниченной
			\item $\{v_k\}$ - бесконечно малая последовательность с ограниченным изменением
			
			Тогда ряд (34) сходится
		\end{enumerate}
		
		\item Пусть выполняются следующие условия \begin{enumerate}
			\item ряд (35) сходится
			\item $\{v_k\}$ - последовательность с ограниченным изменением
			
			Тогда ряд (34) сходится
		\end{enumerate}
	\end{enumerate}
\end{theorem}
\textbf{Доказательство}

\begin{enumerate}
	\item Выпишем оценки, вытекающие из условий первого признака. По условию существует число $M > 0$ такое, что последовательность частичных сумм $\{S_n\}$ ряда (35) удовлетворяет условию $|S_n| \leqslant M$. Поскольку последовательность $v_k$ сходится к нулю и имеет ограниченное изменение, то
	
	\begin{equation}
		\begin{gathered}
			\forall\varepsilon > 0\ \exists N\colon \forall n \geqslant N \ \forall p \in \mathbb{N} \\
			|v_n| < \frac{\varepsilon}{3M} \\
			\displaystyle\sum_{k = n + 1}^{n + p} |v_{k + 1} - v_k| < \frac{\varepsilon}{3M}
		\end{gathered}
	\end{equation}
	
	Для доказательства сходимости ряда (34) воспользуемся критерием Коши и выписанными оценками. Применим тождество Абеля (33), от модуля суммы перейдём к сумме модулей, в каждом слагаемом оценим $|S_n| \leqslant M$ и используем оценки (36). При $n \geqslant N$
	
	\begin{equation*}
		\begin{gathered}
			\left| \displaystyle\sum_{k = n + 1}^{n + p}u_kv_k \right| \leqslant \left| \displaystyle\sum_{k = n + 1}^{n + p - 1}S_k(v_k - v_{k + 1}) \right| + |S_{n + p}v_{n + p}| + |S_nv_{n + 1}| \leqslant M\left( \displaystyle\sum_{k = n + 1}^{n + p - 1}|(v_k - v_{k + 1})| + |v_{n + p}| + |v_{n + 1}| \right) < \\
			< M \left( \frac{\varepsilon}{3M} +  \frac{\varepsilon}{3M} + \frac{\varepsilon}{3M} \right) = \varepsilon
		\end{gathered}
	\end{equation*}
	
	Таким образом, ряд (34) сходится
	
	\item Выпишем оценки вытекающие из условий второго признаква. Так как ряд (35) сходится, то
	
	\begin{equation*}
		\exists M > 0 \colon \{S_n\} \colon |S_n| \leqslant M
	\end{equation*}
	
	Так как последовательность $\{v_k\}$ имеет ограниченное изменение, то
	
	\begin{equation*}
		\forall \varepsilon > 0\ \exists N \colon \forall n \geqslant N \forall p \in \mathbb{N} \quad \displaystyle\sum_{k = n + 1}^{n + p} |v_{k + 1} - v_k| < \frac{\varepsilon}{3M}
	\end{equation*}
	
	Обозначим сумму ряда (35) через $S$, а предел последовательности $\{v_k\}$ через $v$. Но тогда $S_{n + p}v_{n + p} - S_{n - 1}v_n \rightarrow Sv - Sv = 0$, $\exists N \colon \forall n \geqslant N \forall p \in \mathbb{N}$
	
	\begin{equation}
		|S_{n + p}v_{n + p} - S_{n - 1}v_n| < \frac{2\varepsilon}{3}
	\end{equation}
	
	Для доказательства сходимости ряда (34) также воспользуемся критерием Коши и выписанными оценками. Применим тождество Абеля (33), от модуля суммы перейдём к сумме модулей, в первом слагаемом оценим $|S_n| \leqslant M$, далее используем оценки (36, вторая часть), (37). При $n \geqslant N$
	
	\begin{equation*}
		\begin{gathered}
			\left| \displaystyle\sum_{k = n + 1}^{n + p} u_kv_k \right| \leqslant \left| \displaystyle\sum_{k = n + 1}^{n + p} S_k(v_k - v_{k + 1}) \right| + |S_{n + p}v_{n + p} - S_nv_{n + 1}| \leqslant \\
			\leqslant M\left( \displaystyle\sum_{k = n + 1}^{n + p} |(v_k - v_{k + 1})| \right) + |S_{n + p}v_{n + p} - S_nv_{n + 1}| < M\frac{\varepsilon}{3M} + \frac{2\varepsilon}{3} = \varepsilon
		\end{gathered}
	\end{equation*}
	
	Таким образом, ряд (34) сходится.
\end{enumerate}

\begin{flushright}
	$\blacksquare$
\end{flushright}