\section{Признаки Абеля равномерной сходимости функциональных рядов}

\begin{definition}
	Ряды и последовательности, членами которых являются не числа, а функции, определённые на некотором фиксированном множестве, называются функциональными.
\end{definition}

Зафиксируем некоторое множество $\{x\}$ в $m$-мерном евклидовом пространстве $E^m, m \geqslant 1$. Элементами этого множества являются точки $x = (x_1, x_2, ..., x_m)$ с координатами $x_1, x_2, ..., x_m$.

Рассмотрим на некотором множестве $\{x\}$ функциональный ряд вида

\begin{equation}
	\displaystyle\sum_{k = 1}^\infty u_k(x) v_k(x)
\end{equation}

Докажем для этого ряда два признака равномерной сходимости.

Обозначим через $\{S_n(x)\}$ последовательность частичных сумм ряда

\begin{equation}
	\displaystyle\sum_{k  = 1}^\infty u_k(x)
\end{equation}

\begin{theorem}
	(\textbf{два признака Абеля}) \begin{enumerate}
		\item Пусть последовательность $\{S_n(x)\}$ равномерно ограничена на множестве $\{x\}$, а функциональная последовательность $\{v_n(x)\}$ - бесконечно малая с равномерно ограниченным на множестве $\{x\}$ изменением. Тогда ряд (75) сходится равномерно на множестве $\{x\}$.
		
		\item Пусть функциональный ряд (75) сходится равномерно на множестве $\{x\}$, а функциональная последовательность $\{v_n(x)\}$ имеет равномерно ограниченное на множестве $\{x\}$ изменениеи равномерно ограничена на этом множестве. Тогда ряд (75) сходится равномерно на множестве $\{x\}$.
	\end{enumerate}
\end{theorem}
\textbf{Доказательство}

\begin{enumerate}
	\item По условию существует число $M > 0$ такое, что последовательность $\{S_n(x)\}$ частичных сумм ряда (75) для всех номеров $n$ и точек $x$ из множества $\{x\}$ удовлетворяет неравенству $|S_n(x)| \leqslant M$.
	
	Зафиксируем произвольное $\varepsilon > 0$ и по нему номер $N$ такой, что для всех $n$, превосходящих $N$, всех натуральных $p$ и всех точек $x$ множества $\{x\}$ справедливы неравенства
	
	\begin{equation}
		|v_n(x)| < \frac{\varepsilon}{3M}
	\end{equation}
	
	\begin{equation}
		\displaystyle\sum_{k = n + 1}^{n + p} |v_{k + 1}(x) - v_k(x)| < \frac{\varepsilon}{3M}
	\end{equation}
	
	$\{v_n(x)\}$ равномерно сходится к нулю на множестве $\{x\}$ и $\displaystyle\sum_{k = 1}^\infty |v_{k + 1}(x) - v_k(x)|$ равномерно сходится на множестве $\{x\}$.
	
	В силу тождества Абеля
	
	\begin{equation}
		\displaystyle\sum_{k = n + 1}^{n + p} u_kv_k = \displaystyle\sum_{k = n + 1}^{n + p - 1} S_k(v_k - v_{k + 1}) + S_{n + p}v_{n + p} - S_nv_{n + 1},\quad p \neq 1
	\end{equation}
	
	(при $p = 1$ первая сумма в правой части будет заменена нулём). И в силу того, что модуль суммы трёх величин не превосходит сумму трёх модулей, получим
	
	\begin{equation*}
		\left| \displaystyle\sum_{k = n + 1}^{n + p} u_k(x)v_k(x) \right| \leqslant \left| \displaystyle\sum_{k = n + 1}^{n + p - 1} S_k(x)(v_k(x) - v_{k + 1}(x)) \right| + |S_{n + p}(x)||v_{n + p}(x)| + |S_n(x)||v_{n + 1}(x)|
	\end{equation*}
	
	Учитывая, что для всех номеров $n$ и всех точек $x$ из множества $\{x\}$ справедливо неравенство $|S_n(x)| \leqslant M$, получим
	
	\begin{equation*}
		\left| \displaystyle\sum_{k = n + 1}^{n + p} u_k(x)v_k(x) \right| \leqslant M\displaystyle\sum_{k = n + 1}^{n + p} |v_k(x) - v_{k + 1}(x)| + M|v_{n + p}(x)| + M|v_{n + 1}(x)|
	\end{equation*}
	
	Полагая $n > N$, из последнего неравенства и оценок (77) и (78) получим неравенство
	
	\begin{equation*}
		\left| \displaystyle\sum_{k = n + 1}^{n + p} u_k(x)v_k(x) \right| < M\left( \frac{\varepsilon}{3M} + \frac{\varepsilon}{3M} + \frac{\varepsilon}{3M} \right) = \varepsilon
	\end{equation*}
	
	справедливое для всех номеров $n$, превосходящих $N$, всех натуральных $p$ и всех точек $x$ множества $\{x\}$, а это и означает, что ряд (75) сходится равномерно на множестве $\{x\}$ в силу теоремы Коши.
	
	\item Так как функциональная последовательность $\{v_n(x)\}$ равномерно ограничена на множестве $\{x\}$, то существует число $M > 0$ такое, что для всех номеров $n$ и всех $x$ из множества $\{x\}$ справедливо неравенство
	
	\begin{equation}
		|v_n(x)| \leqslant M
	\end{equation}
	
	При условиях второго признака Абеля последовательность частичных сумм $\{S_n(x)\}$ ряда (76) может не быть равномерно ограниченной на множестве $\{x\}$. Поэтому рассмотрим модифицированные суммы
	
	\begin{equation*}
		\hat{S}_n(x) = \displaystyle\sum_{k = N}^n u_k(x),\quad n \geqslant N
	\end{equation*}
	
	где номер $N$ выбираем следующим образом. Зафиксируем произвольное число $\varepsilon > 0$. Ряд (76) по условию теоремы сходится равномерно на множестве $\{x\}$, поэтому, согласно теореме Коши, найдётся номер $N$ такой, что для всех номеров $n \geqslant N$ и всех точек $x$ из множества $\{x\}$ справедливо неравенство
	
	\begin{equation}
		\left| \hat{S_n(x)} \right| < \frac{\varepsilon}{3M}
	\end{equation}
	
	Зафиксируем номер $N$.
	
	В силу критерия Коши равномерной на $\{x\}$ сходимости функционального ряда (71) для числа $M > 0$ найдётся номер $N_1 > N$ такой, что при всех $x$ из множества $\{x\}$, для всех $n \geqslant N_1$ и для любого $p = 1, 2, ...$ справедливо неравенство
	
	\begin{equation}
		\displaystyle\sum_{k = n + 1}^{n + p} |v_{k + 1}(x) - v_k(x)| \leqslant M
	\end{equation}
	
	Поскольку для всех номеров $n > N$ справедливо представление $u_n(x) = \hat{S}_n(x) - \hat{S}_{n - 1}(x)$, то обоснование тождества Абеля (79) после замены $S_n$ на $\hat{S}_n$ не изменится и само тождество можно записать в следующем виде
	
	\begin{equation}
		\displaystyle\sum_{k = n + 1}^{n + p} u_kv_k = \displaystyle\sum_{k = n + 1}^{n + p -1} \hat{S}_k (v_k - v_{k + 1}) + \hat{S}_{n + p}v_{n + p} - \hat{S}_nv_{n + 1}
	\end{equation}
	
	(в случае $p = 1$ в правой части (84) первую сумму следует заменить нулём). В силу этого тождества для всех номеров $n$ и $p$ и всех $x$ из множества $\{x\}$ справедливо неравенство
	
	\begin{equation}
		\left| \displaystyle\sum_{k = n + 1}^{n + p} u_k(x)v_k(x) \right| \leqslant\left| \displaystyle\sum_{k = n + 1}^{n + p - 1} \hat{S}_k(x)(v_k(x) - v_{k + 1}(x)) \right| + \left| \hat{S}_{n + p} \right| |v_{n + p}(x)| + \left| \hat{S}_n(x) \right| |v_{n + 1}(x)|
	\end{equation}
	
	Используя в правой части (84) неравенства (80), (81) и (82), мы для всех $n \geqslant N_1$, всех $p = 1, 2, ...$ и всех $x$ из множества $\{x\}$ получим неравенство
	
	\begin{equation*}
		\left| \displaystyle\sum_{k = n + 1}^{n + p} u_k(x) v_k(x) \right| < \varepsilon
	\end{equation*}
	
	которое в силу критерия Коши доказывает равномерную на множестве $\{x\}$ сходимость функционального ряда (75).
\end{enumerate}

\begin{flushright}
	$\blacksquare$
\end{flushright}

\begin{theorem}
	(\textbf{признак Дирихле-Абеля}) Если функциональный ряд (76) обладает равномерно ограниченной на множестве $\{x\}$ последовательностью частичных сумм, а функциональная последовательность $\{v_n(x)\}$ монотоннаи на множестве $\{x\}$ и равномерно на этом множестве сходится к нулю, то функциональный ряд (75) сходится равномерно на множестве $\{x\}$.
\end{theorem}
\textbf{Доказательство}

Достаточно заметить, что из

\begin{statm}
	Всякая монотонная равномерно сходящаяся на множестве $\{x\}$ функциональная последовательность является последовательностью с равномерно ограниченным на этом множестве изменением.
\end{statm}

следует, что последовательность $\{v_n(x)\}$ обладает на множестве $\{x\}$ равномерно ограниченным изменением, поэтому выполняются все условия теоремы для первого признака Абеля.

\begin{flushright}
	$\blacksquare$
\end{flushright}