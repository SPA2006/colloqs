\section{Метод Чезаро суммирования расходящихся рядов}

Рассмотрим числовой ряд

\begin{equation}
	\displaystyle\sum_{k = 1}^\infty u_k
\end{equation}

Мы называли суммой ряда (52) предел $S$ последовательности $\{S_n\}$ его частичных сумм, если этот предел существует.

Рассмотрим вопрос суммирования расходящихся рядов при помощи обобщённых методов.

\begin{definition}
	Метод суммирования назовём регулярным, если сходящийся числовой ряд (52) имеет и обобщённую сумму, причём эта сумма совпадает с обычной суммой этого ряда.
\end{definition}

Пусть ряд (52) имеет обобщённую сумму $U$, а ряд

\begin{equation}
	\displaystyle\sum_{k = 1}^\infty v_k
\end{equation}

имеет обобщённую сумму $V$.

\begin{definition}
	Метод суммирования назовём линейным, если ряд
	
	\begin{equation}
		\displaystyle\sum_{k = 1}^\infty (\alpha u_k + \beta v_k)
	\end{equation}
	
	где $\alpha$ и $\beta$ - произвольные числа, имеет обобщённую сумму $\alpha U + \beta V$.
\end{definition}

Заметим, что метод Чезаро также называют методом средних арифметических.

\begin{definition}
	Говорят, что ряд (52) суммируем методом Чезаро, если существует предел средних арифметических сумм этого ряда
	
	\begin{equation*}
		\alpha_n = \frac{S_1 + S_2 + ... + S_n}{n}
	\end{equation*}
	
	этот предел
	
	\begin{equation}
		\displaystyle\lim_{n \rightarrow \infty} \alpha_n = S
	\end{equation}
	
	называется обобщённой в смысле Чезаро суммой ряда (52).
\end{definition}

Линейность метода Чезаро очевидна. Действительно, обозначим частичные суммы, обобщённую сумму и последовательность средних арифметических частичных сумм ряда (52) через $U-k, U', \alpha_n'$, те же величины для ряда (53) - через $V_k, V, \alpha_n''$, а через $S_n, \alpha_n$ - частичные суммы и последовательность средних арифметических сумм ряда (54). Тогда $\alpha_n' \rightarrow U, \alpha_n'' \rightarrow V$ при $n \rightarrow \infty$, а так как $S_n = \alpha U_n + \beta V_n$, то $\alpha_n = \alpha\alpha_n' + \beta\alpha_n'' \rightarrow \alpha U + \beta V$ при $n \rightarrow \infty$.

Регулярность метода суммирования Чезаро немедленно следует мз доказанной леммы 1 о сходимости последовательности средних арифметических $n$ чисел, согласно которой из сходимости последовательности $\{S_n\}$ к числу $S$ при $n \rightarrow \infty$ вытекает существование предела (55) и его равенство тому же числу $S$.

Из упомянутой леммы также следует, что если $\{S_n\}$ - бесконечно большая последовательность, то $\{\alpha_n\}$ - такая же последовательность. Это означает, что метод суммирования Чезаро неприменим к расходящимся рядам с неторицательными членами (напр., $\sum \frac{1}{k}, \sum 1, \sum k$).

Примеры рядов, не сходящихся в обычном смысле, но суммируемых методом Чезаро.

\begin{enumerate}
	\item Рассмотрим расходящийся ряд
	
	\begin{equation*}
		\displaystyle\sum_{k = 1}^\infty (-1)^{k - 1} = 1 - 1 + 1 - 1 + ...
	\end{equation*}
	
	$S_{2n} = 0, S_{2n - 1} = 1\quad =>\quad \alpha_n \colon 1, \frac{1}{2}, \frac{2}{3}, \frac{1}{2}, \frac{3}{5}, \frac{1}{2}, ..., \frac{n}{2n - 1}, \frac{n}{2n}, ...$, следовательно, предел (55) существует равен $\frac{1}{2}$, то есть ряд суммируем методом Чезаро и его обобщённая сумма $S$ (в смысле Чезаро) равна $\frac{1}{2}$.
	
	\item Зафиксируем произвольное число $x \in (0, 2\pi)$ и рассмотрим расходящийся ряд
	
	\begin{equation}
		\displaystyle\sum_{k = 1}^\infty = \cos x + \cos 2x + \cos 3x + ...
	\end{equation}
	
	Выражения для $n$-й частичной суммы этого ряда имеет следующий вид
	
	\begin{equation*}
		S_n = \frac{\sin(n + \frac{1}{2})x - \sin\frac{x}{2}}{2\sin\frac{x}{2}}
	\end{equation*}
	
	Посчитаем среднее арифметическое частичных сумм
	
	\begin{equation*}
		\begin{gathered}
			\alpha_n = \frac{S_1 + S_2 + ... + S_n}{n} = \frac{1}{2n\sin\frac{x}{2}} \left( \displaystyle\sum_{l = 1}^n \left( l + \frac{l}{2} \right) x \right) - \frac{1}{2} = \\
			= \frac{1}{4n\sin^2\frac{x}{2}} \displaystyle\sum_{l = 1}^n (\cos lx - \cos (l + 1)x) - \frac{1}{2} = \frac{\cos x - \cos (n + 1)x}{4n\sin^2 \frac{x}{2}} - \frac{1}{2}
		\end{gathered}
	\end{equation*}
	
	Следовательно, $\lim_{n \rightarrow \infty} \alpha_n = -\frac{1}{2}$, то есть ряд (56) суммируем методом Чезаро и его обобщённая сумма (в смысле Чезаро) равна $-\frac{1}{2}$ для каждого значения $x \in (0, 2\pi)$.
	
	\item Рассмотрим расходящийся ряд
	
	\begin{equation}
		\displaystyle\sum_{k = 1}^\infty (-1)^{k - 1}k = 1 - 2 + 3 - 4 + ...
	\end{equation}
	
	Частичные суммы принимают значения $1, -1, 2, -2, ..., n, -n, ...$. Последовательность $\alpha_n$ их средних арифметических раходится, так как имеет две предельные точки $0$ и $\frac{1}{2}$. Таким образом, ряд не суммируем методом Чезаро.
\end{enumerate}