\section{Теорема Римана о перестановке членов в числовых рядах}

\textbf{Понятия абсолютно и условно сходящихся рядов}

Изучим теперь свойства рядов, члены которых являются вещественными числами любого знака

\begin{equation}
	\displaystyle\sum_{k = 1}^\infty u_k,\quad u_k \in \mathbb{R}
\end{equation}

\begin{definition}
	Будем называть ряд (20) абсолютно сходящимся, е  сли сходится ряд
	
	\begin{equation}
		\displaystyle\sum_{k = 1}^\infty |u_k|
	\end{equation}
	
	составленный из модулей ряда (20).
\end{definition}

\begin{statm}
	Из абсолютной сходимости ряда (20) вытекает его сходимость. Обратное неверно.
\end{statm}

\textbf{Доказательство}

Зафиксируем произвольные числа $n, p\in \mathbb{R}$ и запишем очевидное  неравенство для суммы конченого числа  слагаемых

\begin{equation}
	\left| \displaystyle\sum_{k = n + 1}^{n + p} u_k \right| \leqslant \displaystyle\sum_{k = n + 1}^{n + p} |u_k|
\end{equation}

Возьмём  любое число $\varepsilon > 0$. Поскольку ряд (21) сходится, то по критерию Коши сходимости числовых рядов (теорема 1)

\begin{equation*}
	\forall\varepsilon > 0\ \exists N: \forall n \geqslant N\ \forall p \in \mathbb{N} \quad \displaystyle\sum_{k = n + 1}^{n + p} |u_k| < \varepsilon
\end{equation*}

Для того чтобы доказать, что из сходимости ряда (20) не следует сходимость ряда (21), достаточно привести соответсвтующий пример. Рассмотрим ряд

\begin{equation}
	\displaystyle\sum_{k =1}^\infty \frac{(-1)^{k - 1}}{k} = 1 - \frac{1}{2} + \frac{1}{3} - \frac{1}{4} + ...
\end{equation}

Рядом из модулей, соответсвующим ряду (23), является гармонический ряд, расходимость которого мы доказали тремя способами. Так что ряд (23) не сходился абсолютно. Докажем, что этот ряд сходится и его сумма равна $ln 2$. Для каждого значения $x \in [0, 1]$ справедлива формула Маклорена для функции $\ln (1 +  x)$

\begin{equation*}
	\ln (1 + x) = x - \frac{x^2}{2} + \frac{x^3}{3} - \frac{x^4}{4} + ... + (-1)^{n - 1}\frac{x^n}{n} + R_{n + 1}(x)
\end{equation*}

и для остаточного члена $R_{n + 1}$ имеет место оценка 

\begin{equation*}
	|R_{n + 1}| \leqslant \frac{1}{n + 1}\quad x \in [0, 1]
\end{equation*}

Полагая в этих формулах $x = 1$, получим

\begin{equation}
	\ln 2 = 1 - \frac{1}{2} + \frac{1}{3} - \frac{1}{4} + ... + \frac{(-1)^{n - 1}}{n} + R_{n + 1}(1),\quad |R_{n + 1}(1)| \leqslant \frac{1}{n + 1}
\end{equation}

Обозначим через $S_n$ сумму первых $n$ слагаемых в правой части равенства (24). Тогда можем записать

\begin{equation*}
	|S_n - \ln 2| = |R_{n + 1}(1)| \leqslant \frac{1}{n + 1}
\end{equation*}

Так как {\small $\frac{1}{n + 1} \rightarrow 0, n\rightarrow \infty$}, то из последнего неравенства следует, что существует предел {\small $\displaystyle\lim_{n \rightarrow\infty} S_n = \ln 2$}. Поскольку последовательность $\{S_n\}$ является  последовательностью частичных сумм ряда (23), это означает, что ряд (23) сходится и его сумма равна $\ln 2$.

\begin{flushright}
	$\blacksquare$
\end{flushright}

\begin{definition}
	Будем называть ряд (20) условно сходящимся, если он  сходится, а ряд (21) из модулей членов расходится.
\end{definition}

\begin{theorem}
	(теорема Римана) Если ряд (20) сходился условно, то каким бы ни было наперёд заданное вещественное число $L$, можно так переставить члены этого ряда, чтобы преобразованный ряд сходился к числу $L$.
\end{theorem}

\textbf{Доказательство}

Пусть ряд (20) - произвольный условно сходящийся ряд. Изучим его структуру. Из необходимого условия сходимости числового ряда следует, что $\{u_k\}$ является бесконечно малой последовательностью при $k \rightarrow \infty$. Обозначим  через $p_1, p_2, p_3, ...$ неотрицательные члены ряда (20), выписанные в таком же порядке, в каком они стоят в этом ряде, а через $q_1, q_2, q_3, ...$ - модули отрицательных членов ряда (20), выписанные в таком же порядке, в каком они стоят в этом ряде. Ряд (20) содержит бесконечное число как положительных, так и отрицательных членов, ибо если бы членов одного знака было бы конечное число, то, отбросив не влияющее на сходимость конечное число первых членоы, мы бы получили ряд, состоящий из членов одного знака, для которого сходимость означала бы абсолютную сходимость. Так как $\{u_k\}$ является бесконечно малой последовательностью при $k \rightarrow \infty$, то $\{p_k\}$ и $\{q_k\}$ также являются бесконечно малыми последовательностями.

Итак, с рядом (20) связаны два бесконечных ряда с неотрицательными членами {\small $\displaystyle\sum_{k = 1}^\infty p_k$} и {\small $\displaystyle\sum_{k = 1}^\infty q_k$}. Будем обозначать первый из этих рядов символом $P$, а второй - символом $Q$. Докажем, что оба ряда $P$ и $Q$ являются расходящимися. Обозначим $n$-ю частичную сумму ряда (20) символом $S_n$, символом $P_n$ - сумму всех положительных членов, входящих в $S_n$, символом $Q_n$ - сумму всех отрицательных членов, входящих в $S_n$. Тогда, очевидно, $S_n = P_n - Q_n$, и так как по условию ряд (20) сходится к некоторому числу $S$, то

\begin{equation}
	\displaystyle\lim_{n \rightarrow \infty} (P_n - Q_n) = S
\end{equation}

С другой стороны, так как ряд (20) не сходится абсолютно, то

\begin{equation}
	\displaystyle\lim_{n \rightarrow \infty} (P_n + Q_n) = + \infty
\end{equation}

Сопоставляя (25) и (26), получим {\small $\displaystyle\lim_{n \rightarrow \infty} P_n = +\infty, \displaystyle\lim_{n \rightarrow \infty} Q_n = +\infty$}, то есть оба ряда $P$ и $Q$ расходятся.

Из расходимости рядов $P$ и $Q$ вытекает, что даже после удаления конечного числа первых членов этих рядов мы можем взять из оставшихся членов как ряда $P$, так и ряда $Q$ столь большое число членов, что их сумма превзойдёт любое наперёд заданное число. Опираясь на этот факт, докажем, что можно так переставить члены исходного ряда (20), что в результате получится ряд, сходящийсяк наперёд заданному числу $L$.

В самом деле, мы получим требуемый ряд следующим образом. Сначала выберемиз исходного ряда (20) ровно столько неотрицательных членов $p_1, p_2, p_3, ..., p_{k_1}$, чтобы их сумма превзошла $L$

\begin{equation*}
	p_1 + p_2 + p_3 + ... + p_{k_1} > L,\quad (\mbox{при этом } p_1 + p_2 + p_3 + ... + p_{k_1 - 1} \leqslant L)
\end{equation*}

Затем добавим к выбранным членам ровно столько отрицательных членов $-q_1, -q_2, ..., -q_{k_2}$, чтобы общая сумма $p_1 + p_2 + ... + p_{k_1} - q_1 - q_2 - ... - q_{k_2}$  оказалась меньше $L$. Затем снова добавим ровно столько положительных членов $p_{k_1 + 1}, p_{k_1 + 2}, ..., p_{k_3}$, чтобы общая сумма $p_1 + p_2 + ... + p_{k_1} - q_1 - q_2 - ... - q_{k_2} + p_{k_1 + 1} + ... + p_{k_3}$ оказалась больше $L$. Продолжая аналогичные рассуждения далее, мы бесконечный ряд, в состав которого войдут все члены исходного ряда (20), так как каждый раз нам придется добавлять хотя бы один положительный или отрицательный член исходного ряда.

Докажем, что полученный ряд сходится к L. Заметим, что в полученном ряде последовательно чередуются группы неотрицательных и группы отрицательных членов. Если частичная сумма полученного ряда заканчивается полностью завершенной группой, то отклонение этой частичной суммы от числа L не превосходит модуля последнего ее члена (ибо мы добавляем в данную группу члены ровно до тех пор, пока общая сумма не «перейдет» через число L). Если же частичная сумма заканчивается не полностью завершенной группой, то отклонение этой частичной суммы от числа L не превосходит модуля последнего члена предпоследней из групп. Для установления сходимости ряда к L достаточно убедиться в том, что модули последних членов групп образуют бесконечно малую последовательность, а это непосредственно вытекает из необходимого условия сходимости исходного ряда (20).

\begin{flushright}
	$\blacksquare$
\end{flushright}