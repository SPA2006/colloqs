\section{Признаки сравнения}
В этой части будем рассматривать числовые ряды с неотрицательными членами

\begin{equation}
	\displaystyle\sum_{k = 1}^\infty p_k,\quad p_k \geqslant 0
\end{equation}

Сформулируем три утверждения, в которых исследуемый ряд (3) сравнивается с рядом

\begin{equation}
	\displaystyle\sum_{k = 1}^\infty p_k',\qquad p_k' \geqslant 0
\end{equation}

сходимость (или расходимость) которого известна.

\begin{theorem}
	(\textbf{первый признак сравнения}) Пусть для всех номеров $k$ начиная с некоторого $k_0$ выполняется неравенство
	
	\begin{equation}
		p_k \leqslant p_k',\qquad k \geqslant k_0 > 0
	\end{equation}
	
	Тогда из сходимости ряда (4) вытекает сходимость ряда (3), а из расходимости ряда (3) вытекает расходимость ряда (4).
\end{theorem}

\textbf{Доказательство}

Докажем второе утверждение. Если ряд (3) расходится, то и ряд (4) расходится, так как в противном случае из первого утверждения теоремы следовала бы сходимость ряда (3). Для доказательства первого утверждения заметим, что, не ограничивая общности, можно считать неравенство (5) выполненным для всех значений $k = 1, 2, ...$. Но тогда для всех номеров $n$ справедливо неравенство $S_n \leqslant s_n'$, где $\{S-n\},\: \{S_n'\}$ - последовательности частичных сумм рядов (3) и (4) соответственно. Если теперь ряд (4) расходится, то последовательность $\{S_n'\}$ является ограниченной, следовательно, ограниченной является и последовательность $\{S_n\}$; в силу теоремы

\begin{theorem}
	Для того, чтобы ряд (3) сходился, необходимо и достаточно, чтобы последовательность его частичных сумм $\{S_n\}$ была ограниченной.
\end{theorem}

ряд (3) сходится. Теорема доказана.

\begin{flushright}
	$\blacksquare$
\end{flushright}

\begin{note}
	Неравенство (5) в условии теоремы (3) можно заменить на неравенство $p_k \leqslant cp_k'$, где $c$ - положительная постоянная, утверждение теоремы останется в силе (следует из замечения 2 первого билета).
\end{note}

\begin{theorem}
	(\textbf{второй признак сравнения}) Пусть начиная с некоторого номера $k_0$ все члены ряда (4) строго положительны $p_k' > 0, k \geqslant k_0$. Пусть существует конечный и отличный от нуля предел, обозначим его через $L$
	
	\begin{equation}
		\displaystyle\lim_{k \rightarrow \infty} \frac{p_k}{p_k'} = L
	\end{equation}
	
	Тогда оба ряда (3) и (4) сходятся или расходятся одновременно.
\end{theorem}

\textbf{Доказательство}

Из определения предела числовой последовательности получаем

\begin{equation*}
	\forall\varepsilon > 0\ \exists N\in \mathbb{N}: \forall k \geqslant N\ L - \varepsilon < \frac{p_k}{p_k'} < L + \varepsilon
\end{equation*}

или $(L - \varepsilon)p_k' < p_k < (L + \varepsilon)p_k'$. Из последних неравенств и теоремы (3) (полагаем $\varepsilon < L$) следует утверждение теоремы. Теорема доказана.

\begin{flushright}
	$\blacksquare$
\end{flushright}

\begin{theorem}
	(\textbf{третий признак сравнения}) Пусть начиная с некоторого номера $k_0$ все члены каждого из рядов (3), (4) строго положительны и выполняется неравенство
	
	\begin{equation}
		\frac{p_{k + 1}}{p_k} \leqslant \frac{p_{k+1}'}{p_k'}, \qquad k \geqslant k_0 > 0
	\end{equation}
	
	Тогда справедливы утверждения теоремы (3).
\end{theorem}

\textbf{Доказательство}

Не ограничивая общности рассуждений будем считать, что неравенство (7) выполняется для всех номеров $k = 1, 2, ....$ Полагая в (7) последовательно $k = 1, 2, ..., n$, где $n$ - произвольный номер, получим цепочку неравенств

\begin{equation*}
	\frac{p_2}{p_1} \leqslant \frac{p_2'}{p_1'}, \frac{p_3}{p_2} \leqslant \frac{p_3'}{p_2'}, ..., \frac{p_n}{p_{n-1}} \leqslant \frac{p_n'}{p_{n-1}'}
\end{equation*}

Перемножив почленно эти неравенства, получим, что справедливо следующее неравенство

\begin{equation*}
	\frac{p_n}{p_1} \leqslant \frac{p_n'}{p_1'},\quad \mbox{ или }\quad p_n \leqslant \frac{p_1}{p_1'}p_n'\quad (n \geqslant 1)
\end{equation*}

Применяя теперь утверждение теоремы 3 и замечание к этой теореме, получаем утверждение теоремы 6. Теорема доказана.

\begin{flushright}
	$\blacksquare$
\end{flushright}