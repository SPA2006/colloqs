\section{Последовательности с равномерно ограниченным изменением и их свойства}

\setcounter{exmpl}{0}

\begin{definition}
	Последовательность $\{v_k\}$ назовём последовательностью с ограниченным изменением, если сходится числовой ряд
	
	\begin{equation}
		\displaystyle\sum_{k = 1}^\infty |v_{k + 1} - v_k| = |v_2 - v_1| + |v_3 - v_2| + ...
	\end{equation}
	
	Сумму ряда (71) в случае его сходимости обозначим через $V$.
\end{definition}

\begin{definition}
	Число $V$ называется полным изменением последовательности $\{v_k\}$.
\end{definition}

\begin{exmpl}
	Последовательность с общим членом $v_k = k$, очевидно, имеет неограниченное изменение $(v_{k +  1} - v_k = 1)$, и ряд (71) расходится.
\end{exmpl}

\begin{exmpl}
	Последовательность с общим членом $v_k = \frac{1}{k}$ имеет ограниченное изменение
	
	\begin{equation}
		S_n = \displaystyle\sum_{k = 1}^n (v_k - v_{k + 1}) = \left( 1 - \frac{1}{2} \right) + \left( \frac{1}{2} - \frac{1}{3} \right) + ... + \left( \frac{1}{n} - \frac{1}{n + 1} \right) = 1 - \frac{1}{n + 1} \rightarrow 1 \quad \mbox{при} \quad n \rightarrow \infty
	\end{equation}
	
	ряд (71) сходится.
\end{exmpl}

Общий член последовательности с ограниченным изменением не обязан стремиться к нулю.

\begin{exmpl}
	Покажем, что не всякая сходящаяся последовательность имеет ограниченное изменение. Пусть $v_k = \frac{(-1)^{k - 1}}{k}$. Тогда
	
	\begin{equation}
		\begin{gathered}
			S_n = \displaystyle\sum_{k = 1}^n |v_{k + 1} - v_k| = \left| -\frac{1}{2} - 1 \right| + \left| \frac{1}{3} + \frac{1}{2} \right| + \left| -\frac{1}{4} - \frac{1}{3} \right| + ... + \left| (-1)^k \left( \frac{1}{n + 1} + \frac{1}{n} \right) \right| = \\
			= 2\left( 1 + \frac{1}{2} + ... + \frac{1}{n} \right) - 1 + \frac{1}{n + 1} = 2H_n - 1 + \frac{1}{n + 1} \rightarrow +\infty \quad \mbox{при} \quad n \rightarrow \infty
		\end{gathered}
	\end{equation}
	
	ряд (71) расходится, сходящаяся к нулю последовательность $\{v_k\}$ не имеет ограниченного изменения.
\end{exmpl}

\begin{statm}
	Всякая последовательность с ограниченным изменением сходится. Обратное неверно.
\end{statm}
\textbf{Доказательство}

То, что не всякая сходящаяся последовательность имеет ограниченное изменение, следует из рассмотренного примера 3. Заметим далее, что из сходимости ряда (71) вытекает сходимость ряда без модулей.

\begin{equation}
	\displaystyle\sum_{k = 1}^\infty (v_{k + 1} - v_k)
\end{equation}

Обозначим сумму ряда (72) через $S$, а его $n$-ю частичную сумму - через $S_n$. Поскольку $S_n = (v_2 - v_1) + (v_3 - v_2) + ... + (v_{n + 1} - v_n) = v_{n + 1} - v_1$, то $v_{n + 1} = S_n + v_1 \rightarrow S + v_1$ при $n \rightarrow \infty$. Это означает, что последовательность $\{v_n\}$ сходится к пределу $S + v_1$.

\begin{flushright}
	$\blacksquare$
\end{flushright}

\begin{statm}
	Всякая монотонная ограниченная последовательность является последовательностью с ограниченным изменением.
\end{statm}
\textbf{Доказательство}

Так как последовательность $\{v_k\}$ монотонная и ограниченная, то она сходится. Обозначим её предел через $v$. Обозначим $n$-ю частичную сумму ряда (71) через $S_n$. Так как последовательность $\{v_k\}$ монотонна, все разности $v_{k + 1} - v_k$ имеют один знак. Но тогда

\begin{equation*}
	S_n = \displaystyle\sum_{k = 1}^n |v_{k + 1} - v_k| = \left| \displaystyle\sum_{k = 1}^n (v_{k + 1} - v_k) \right| = |v_{n + 1} - v_1| \rightarrow |v - v_1| \quad \mbox{при} \quad n \rightarrow \infty
\end{equation*}

Это означает, что ряд (71) чходится. его сумма равна $V = |v - v_1|$, а последовательность $\{v_k\}$ имеет ограниченное изменение.

\begin{flushright}
	$\blacksquare$
\end{flushright}

\begin{statm}
	Для того чтобы последовательность имела ограниченное изменение, необходимо и достаточно, чтобы её общий член можно было представить в виде разности двух ограниченных неубывающих последовательностей.
\end{statm}