\section{Взаимосвязь между сходимостью четырёх рядов: повторных, двойного и одинарного}

Рассмотрим счётное множество числовых последовательностей $(a_{kl})_{k,l = 1}^\infty$, где $k$ - номер последовательности, а $l$ - номер элемента последовательности. Или, по-другому, рассмотрим матрицу, содержащую бесконечное число строк и бесконечное число столбцов

\begin{equation}
	\begin{pmatrix}
		a_{11} & a_{12} & \cdots & a_{1l} &  \cdots \\
		a_{21} & a_{22} & \cdots & a_{2l} &  \cdots \\
		\cdots & \cdots & \cdots & \cdots & \cdots \\
		a_{k1} & a_{k2} & \cdots & a_{kl} &  \cdots \\
		\cdots & \cdots & \cdots & \cdots & \cdots \\
	\end{pmatrix}
\end{equation}

Приводя формальное суммирование элементов матрицы (44), можно составить из неё различные ряды.

Если сначала просуммировать каждую строку матрицы (44) отдельно, то получим бесконечную последовательность рядов вида

\begin{equation}
	\displaystyle\sum_{l = 1}^\infty a_{kl},\quad k = 1, 2, ...
\end{equation}

Просуммировав эту последовательность, получим формальную сумму

\begin{equation}
	\displaystyle\sum_{k = 1}^\infty\left( \displaystyle\sum_{l = 1}^\infty a_{kl} \right)
\end{equation}

Эту сумму называют повторным рядом.

\begin{definition}
	Повторный ряд (46) называется сходящимся, если сходится каждый из рядов (45) и если сходится ряд {\small $\displaystyle\sum_{k = 1}^\infty A_k$}, в котором $A_k$ обозначает сумму $k$-го рдяа (45).
\end{definition}

Другой повторный ряд 

\begin{equation}
	\displaystyle\sum_{l = 1}^\infty \left( \displaystyle\sum_{k = 1}^\infty a_{kl} \right)
\end{equation}

получается, если сначала просуммировать отдельно каждый столбец матрицы (44), а затем взять сумму элементов полученной при этом последовательности. Сходимсоть ряда (47) определяется аналогично сходимости ряда (46).

С матрицей (44) кроме повторных рядов (46), (47) связывают ещё так называемый двойной ряд

\begin{equation}
	\displaystyle\sum_{k,l = 1}^\infty a_{kl}
\end{equation}

\begin{definition}
	Двойной ряд (48) называется сходящимся, если при независимом стремлении двух индексов $m$ и $n$ к бесконечности существет конечный предел
	
	\begin{equation}
		S = \displaystyle\lim_{\substack{m \rightarrow \infty \\ 
																		n \rightarrow \infty}} S_{mn}
	\end{equation}
	
	так называемых прямоугольных частичных сумм
	
	\begin{equation*}
		S_{mn} = \displaystyle\sum_{k = 1}^m \displaystyle\sum_{l = 1}^n a_{kl}
	\end{equation*}
	
	При этом предел $S$ называется суммой двойного ряда (48).
\end{definition}

Из этого определения сразу следует, что если двойной ряд (48) получен посредством перемножения двух сходящихся "одинарных" рядов {\small $\displaystyle\sum_{k = 1}^\infty b_k$} и {\small $\displaystyle\sum_{l = 1}^\infty c_l$}, то есть члены двойного ряда (48) равны $a_{kl}=b_kc_l$, то этот двойной ряд сходится и его сумма равна произведению сумм этих рядов

\begin{equation*}
	S_{mn} = \displaystyle\sum_{k = 1}^m \displaystyle\sum_{l = 1}^n b_kc_l = \left( \displaystyle\sum_{k = 1}^m b_k \right) \left( \displaystyle\sum_{l = 1}^n c_l \right) \rightarrow \left( \displaystyle\sum_{k = 1}^\infty b_k \right) \left( \displaystyle\sum_{l = 1}^\infty c_l \right) = S, \mbox{ при } m,n \rightarrow \infty
\end{equation*}

\begin{statm}
	Необходимое условие сходимости двойного ряда (48) - стремление к нулю его общего члена, то есть существование равного нулю предела $\displaystyle\lim_{m, n \rightarrow \infty} a_{mn}$ при независимом стремлении $m$ и $n$ к бесконечности.
\end{statm}
\textbf{Доказательство}

Сразу вытекает из представления общего члена ряда (48) через прямоугольные частичные суммы

\begin{equation*}
	a_{mn} = S_{mn} - S_{m, n-1} - S_{m-1, n} + S_{m-1, n-1} \rightarrow S - S - (S - S) = 0 \mbox{ при } m,n \rightarrow \infty
\end{equation*}

Исследуем, как связаны между собой сходимость двойного ряда и сходимость повторного ряда.

\begin{theorem}
	Если сходится двойной ряд (48) и если сходятся все ряды по строкам (45), то сходится и повторный ряд (46), причём к той же сумме, к которой сходится и двойной ряд (48).
	
	Аналогичное утверждение справделиво для повторного ряда (47).
\end{theorem}
\textbf{Доказательство}

Определим сходимость повторного ряда (46) в терминах последовательности $\{S_{mn}\}$. Так как суммы рядов (45) равны $A_k$ для каждого $k$, то устрпемляя в $S_{mn}$ индекс $n$ к бесконечности, получим

\begin{equation*}
	\displaystyle\lim S_{mn} = \displaystyle\sum_{k  =1}^m A_k
\end{equation*}

Обозначим этот предел через $\varphi_m$. Сумма повторного ряда (46) определяется как предел последовательности $\{\varphi_m\}$ при $m \rightarrow \infty$, то есть это повторный предел

\begin{equation*}
	\displaystyle\lim_{m \rightarrow \infty} \varphi_m = \displaystyle\lim_{m \rightarrow \infty} \left( \displaystyle\lim_{n \rightarrow \infty} S_{mn} \right)
\end{equation*}

Таким образом, нузжно доказать существование указанного двойного повторного предела в предположении существования двойного предела прямоугольны хчастичных сумм $S_{mn}$ и существования предела этих сумм при $n \rightarrow \infty$ при каждом значении $m$, а также доказать, что эти двойной и повторный интегралы равны между собой.

По условию двойной ряд (48) сходится. Обозначим его сумму через $S$. Запишем определение двойного предела, равного $S$

\begin{equation*}
	\forall \varepsilon > 0\ \exists m_0, n_0 \in \mathbb{N} \colon \forall m \geqslant m_0, n\geqslant n_0 \quad |S_{mn} - S| < \frac{\varepsilon}{2}
\end{equation*}

то есть для указанных значений $m$ и $n$

\begin{equation*}
	S - \frac{\varepsilon}{2} < S_{mn} < S + \frac{\varepsilon}{2}
\end{equation*}


Устремляя $n$ к бесконечности, получим, что
 
 \begin{equation*}
 	\forall \varepsilon > 0 \ \exists m_0 \in \mathbb{N} \colon \forall m \geqslant m_0 \quad |\varphi_m - S| \leqslant \frac{\varepsilon}{2} < \varepsilon
 \end{equation*}
 
 а это означает, что существует предел
 
 \begin{equation*}
 	\displaystyle\lim_{m \rightarrow \infty} \varphi_m = \displaystyle\lim_{m \rightarrow \infty} \displaystyle\lim_{n \rightarrow \infty} S_{mn} = S
 \end{equation*}
 
 \begin{flushright}
 	$\blacksquare$
 \end{flushright}
 
 \begin{note}
 	Рассмотрим пример показывающий, что от равномерной сходимости по строкам нельзя отказаться
 \end{note}
 
 \begin{exmpl}
 	Рассмотрим следующую бесконечную матрицу
 	
 	\begin{equation*}
 		\begin{pmatrix}
 			1 & -1 & 1 & -1 & \cdots \\
 			-1 & 1 & -1 & 1 & \cdots \\
 			\frac{1}{2} & -\frac{1}{2} & \frac{1}{2} & -\frac{1}{2} & \cdots \\
 			-\frac{1}{2} & \frac{1}{2} & -\frac{1}{2} & \frac{1}{2} & \cdots \\
 			\frac{1}{3} & -\frac{1}{3} & \frac{1}{3} & -\frac{1}{3} & \cdots \\
 			\cdots & \cdots & \cdots & \cdots & \cdots
 		\end{pmatrix}
 	\end{equation*}
 	
 	Двойной ряд для этой матрицы сходится, его сумма равна нулю, так как прямоугольные частичные суммы $S_{mn}$ в этом случае равны либо нулю, либо $a_{mn}$ - элементу в правом нижнем углу соответствующего "прямоугольника" из элементов $(a_{kl}), 1 \leqslant m, 1 \leqslant l \leqslant n$. А из структуры матрицы следует, что $a_{mn} \rightarrow 0$ при $m, n \rightarrow \infty$. Между тем все ряды по строкам этой матрицы расходятся, то есть двойной ряд (48) сходится, а повторный ряд (46) расходится.
 	
 	Далее заметим, что все ряды по столбцам этой матрицы сходятся, их суммы равны нулю, следовательно, и повторный ряд (47) сходится и имеет сумму, равную нулю.
 \end{exmpl}
 
 \begin{note}
 	В теоерме () (последней, \textbf{будет изменено}) и примере 1 двойной ряд (48) сходился. Связана ли сходимость повторного ряда (46) или (47) со сходимостью двойного ряда (48)? Следующий пеример показывает, что повторные ряды (46) и (47) могут сходиться и в случае расходящегося двойного ряда (48).
 \end{note}
 
 \begin{exmpl}
 	Рассмотрим слекдующую бесконечную матрицу $(a_{kl})$
 	
 	\begin{equation*}
 		\begin{pmatrix}
 			1 & -1 & 0 & 0 & 0 & \cdots \\
 			0 & 1 & -1 & 0 & 0 & \cdots \\
 			0 & 0 & 1 & -1 & 0 & \cdots \\
 			\cdots & \cdots & \cdots & \cdots & \cdots & \cdots
 		\end{pmatrix}
 	\end{equation*}
 	
 	Прямоугольные частичные суммы $S_{mn}$, отвечающие этой матрице, принимают всего два значения - 0 и 1. Причём каждое значение принимается бесконечно много раз, двойного предела не существует, двойной ряд (48) расходится.
 \end{exmpl}
 
 \begin{theorem}
 	Если все элементы матрицы (44) неотрицательны, то для составленного из этой матрицы двойного ряда (48) необходимо и достаточно, чтобы его частичные суммы $\{S_{mn}\}$ были ограниченными.
 \end{theorem}
 \textbf{Доказательство}
 
 Необходимость очевидна: если обозначить через $S$ сумму ряда (48), то $S_{mn} \leqslant S$ для всех номеров $m$ и $n$. Для доказательства достаточности заметим, что из ограниченности множества частичных сумм $\{S_{mn}\}$ вытекает существование точной верхней грани этого множества, которую обозначим через $S$
 
 \begin{equation*}
 	S = \displaystyle\sup_{1 \leqslant m, n < \infty} S_{mn}
 \end{equation*}
 
 По определению точной верхней грани
 
 \begin{equation*}
 	\begin{gathered}
 		\forall \varepsilon > 0 \ \exists S_{m_0n_0} \colon S - \varepsilon < S_{m_0n_0} \leqslant S;\ \forall m, n \colon m \leqslant m_0, n \leqslant n_0 \ S_{mn} \geqslant S_{m_0n_0}\ => \\
 		S - \varepsilon < S_{m_0n_0} \leqslant S_{mn} \leqslant S\ \forall m \geqslant m_0, n \geqslant n_0
 	\end{gathered}
 \end{equation*}
 
 это и означает существование двойного предела
 
 \begin{equation*}
 	\displaystyle\lim_{m,n \rightarrow \infty} S_{mn} = S
 \end{equation*}
 
 то есть сходимость двойного ряда (48).
 
 \begin{flushright}
 	$\blacksquare$
 \end{flushright}
 
 \begin{definition}
 	Двойной ряд (48) называется абсолютно сходящимся, если сходится двойной ряд
 	
 	\begin{equation}
 		\displaystyle\sum_{k,l = 1}^\infty |a_{kl}|
 	\end{equation}
 	
 	составленный из модулей элепментов матрицы (44).
 \end{definition}
 
 \begin{theorem}
 	Если сходится двойной ряд (48) из модулей, то сходится и двойной ряд (48).
 \end{theorem}
 \textbf{Доказательство}
 
 Из сходимости ряда () следует ограниченность частичных сумм этого ряда (по теореме ). Положим
 
 \begin{equation*}
 	p_{kl} = \frac{|a_{kl}| + a_{kl}}{2},\quad q_{kl} = \frac{|a_{kl}| - a_{kl}}{2}
 \end{equation*}
 
 тогда из очевидных неравенств $0 \leqslant p_{kl}, q_{kl} \leqslant |a_{kl}|$ следует ограниченность частичных сумм каждого из рядов
 
 \begin{equation*}
 	\displaystyle\sum_{k,l = 1}^\infty p_{kl},\ \displaystyle\sum_{k,l = 1}^\infty q_{kl}
 \end{equation*}
 
 В силу теоремы (20) эти ряды сходятся. Обозначим их суммы соответственно через $P$ и $Q$. Но тогда из соотношения $a_{kl} = p_{kl} - q_{kl}$ следует, что двойной ряд (42) сходится к сумме $P - Q$.
 
 \begin{flushright}
 	$\blacksquare$
 \end{flushright}
 
 Далее будем рассматривать и обычный ряд
 
 \begin{equation}
 	\displaystyle\sum_{r = 1}^\infty a_r
 \end{equation}
 
 членами которого являются занумерованные в произвольном порядке элементы матрицы (44). Докажем теорему о взаимосвязи между сходимостью четырёх рядов, связанных с матрицей (44), - двух повторных, двойного и одинарного.
 
 \begin{equation*}
 	\begin{gathered}
 		\displaystyle\sum_{k = 1}^\infty a_{kl},\ (2)\ k = 1, 2, ..., \qquad \quad \displaystyle\sum_{k = 1}^\infty |a_{kl}| \ (2') \\
 		\displaystyle\sum_{k = 1}^\infty \displaystyle\sum_{l = 1}^\infty a_{kl}, \ (3) \qquad \qquad \qquad \qquad\ \displaystyle\sum_{k = 1}^\infty\displaystyle\sum_{l = 1}^\infty |a_{kl}| \ (3') \\
 		\displaystyle\sum_{l = 1}^\infty \displaystyle\sum_{k = 1}^\infty a_{kl}, \ (4) \qquad \qquad \qquad \qquad\ \displaystyle\sum_{l = 1}^\infty\displaystyle\sum_{k = 1}^\infty |a_{kl}| \ (4') \\
 		\displaystyle\sum_{k,l = 1}^\infty a_{kl},\ (5) \qquad \qquad \qquad \quad\ \ \ \displaystyle\sum_{k,l = 1}^\infty |a_{kl}| \ (5') \\
 		\displaystyle\sum_{r = 1}^\infty a_r, \ (6) \qquad \qquad \qquad \qquad\ \ \ \displaystyle\sum_{r = 1}^\infty |a_r|\ (6')
 	\end{gathered}
 \end{equation*}
 
 \begin{theorem}
 	Рассмотрим четыре ряда (3)-(6). Если сходится хотя бы один из рядов с модулями (3')-(6'), то сходятся все ряды (3)-(6) и суммы их равны между собой.
 \end{theorem}
 \textbf{Доказательство}
 
 Сначала докажем, что из сходимости одного из рядов (3')-(6') вытекает сходимость всех рядов (3)-(6). Поскольку для повторных рядов (3) и (4) рассуждения аналогичны (изменяется только порядок следования индексов), то далее рассматриваем только ряд (3).
 
 Докажем сходимости рядов по следующей цепочке
 
 \begin{equation*}
 	(3') \rightarrow (6'), (6) \rightarrow (5'), (5) \rightarrow (3'), (3)
 \end{equation*}
 
 \begin{enumerate}
 	\item $(3') \rightarrow (6')$. Пусть сходится повторный ряд $(3')$. Обозначим сумму этого ряда через $S'$, его частичные суммы через $S_{mn}'$, а $p$-ю частичную сумму ряда $(6')$ через $S_p'$. Для всех номеров $m$ и $n$ справедливо неравенство $S_{mn}' \leqslant S'$. Зафиксируем любой номер $p$. Найдутся столь большие номера $m$ и $n$, что все члены ряда (6), входящие в $p$-ю частичную сумму этого ряда, будут содержаться в первых $m$ строках и $n$ столбцах матрицы (44). Но тогда справедливо неравенство $S_p' \leqslant S_{mn}' \leqslant S'$, т.е. $S_p' \leqslant S'$ (верное для любого номера $p$); все частичные суммы ряда с неотрицательными членами $(6')$ ограничены и по теореме  ряд $(6')$ сходится. Из абсолютной сходимости ряда $(6)$ следует его сходимость.
 	
 	\item $(6') \rightarrow (5'), (5)$. Пусть сходится ряд $(6')$. Тогда последовательность его частичных сумм $\{S_p'\}$ ограничена: найдётся такая постоянная $M > 0$, что $S_p' \leqslant M$ для всех номеров. Зафиксируем любую частичную сумму $S_{mn}'$ ряда $(5')$. Найдётся такой большой номер $p$, что $p$-я частичная сумма ряда $(6)$ будет содержать все чдены, входящие в частичную сумму $S_{mn}$ ряда $(5)$. Но тогда справедливы неравенства $S_{mn} \leqslant S_p' \leqslant M$, т.е. $S_{mn}' \leqslant M$, верное для всех номеров $m$ и $n$. Следовательно, множество $\{S_{mn}'\}$ ограничено и ряд $(5')$ сходится (согласно теореме ). Из теоремы  в свою очередь вытекает сходимость двойного ряда $(5)$.
 	
 	\item $(5') \rightarrow (3'), (3)$. Пусть сходится двойной ряд $(5')$. Из теоремы 16 следует сходимость ряда $(5)$. Для доказательства сходимости повторных рядов $(3')$ и $(3)$ в силу теоремы 15 достаточно доказать сходимость каждого из рядов $(2')$ и $(2)$. Сходимость ряда $(2)$ при каждом $k$ вытекает из сходимости соответствующего ряда $(2')$. Для сходимости рядов $(2')$ по теореме 2 достаточно доказать, что каждый из этих рядов имеет ограниченную последовательность частичных сумм. Но это очевидно, поскольку каждая из этих сумм ограничена суммой двойного ряда $(5')$. Из сходимости рядов $(5')$ и $(2')$ следует сходимость ряда $(3')$ (и равенство сумм двойного и повторного рядов), из сходимости рядов $(5)$ и $(2)$ следует сходимость ряда $(3)$ и равенство его суммы сумме ряда $(5)$.
 	
 	Докажем, что суммы рядов $(5)$ и $(6)$ также совпадают. Пусть $S$ - сумма двойного ряда $(5)$. Очевидно, что и сумма ряда $(6)$ равна $S$, так как в силу абсолютной схолимости этого ряда его сумма не меняется при изменении порядка следования его членов и этот порядок можно изменить так, что частичные суммы после изменения порядка будут содержать в качестве подмножества частичные суммы $S_{mn}$ двойного ряда $(5)$.
 	
 	\begin{flushright}
 		$\blacksquare$
 	\end{flushright}
 \end{enumerate}
 
 \begin{note}
 	Из доказательства теоремы 17 следует, что при выполнении условия этой теоремы сходятся все ряды $(3')-(6')$ и имеют суммы, равные между собой.
 \end{note}