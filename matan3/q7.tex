\section{Последовательности с ограниченным изменением и их свойства}

\begin{definition}
	Последовательность $\{a_n\}$ назовём последовательностью с ограниченным изменением, если сходится числовой ряд
	
	\begin{equation}
		\displaystyle\sum_{k = 1}^\infty |v_{k + 1} - v_k| = |v_2 - v_1| + |v_3 - v_2| + ...
	\end{equation}
	
	Сумму ряда (31) в случае его сходимости обозначим через $V$. Число $V$ называется полным изменением последовательности $\{v_k\}$.
\end{definition}

Сформулируем необходимое условие для последовательностей с ограниченным изменением.

\begin{statm}
	Всякая последовательность с ограниченным изменением сходится. Обратное неверно.
\end{statm}
\textbf{Доказательство}

Не всякая сходящаяся последовательность имеет ограниченное изменение. Заметим далее, что из сходимости ряда (31) вытекает сходимость ряда без модулей

\begin{equation*}
	\displaystyle\sum_{k = 1}^\infty (v_{k + 1} - v_k)
\end{equation*}

Обозначим сумму ряда (32) через $S$, а его $n$-ю частичную сумму - через $S_n$. Поскольку $S_n = (v_2 - v_1) + (v_3 - v_2) + ... + (v_{n + 1} - v_n) = v_{n + 1} - v_1$, то $v_{n + 1} = S_n + v_1 \rightarrow S - v_1$ при $n \rightarrow \infty$. Это означает, что последовательность $\{v_n\}$ сходится к пределу $S + v_1$.

\begin{flushright}
	$\blacksquare$
\end{flushright}

Сформулируем теперь достаточное условие того, чтобы последовательность имела ограниченное изменение.

\begin{statm}
	Всякая монотонная ограниченная последовательность является последовательностью с ограниченным изменением.
\end{statm}
\textbf{Доказательство}

Так как последовательность $\{v_k\}$ монотонная и ограниченна, то она сходится. Обозначим $n$-ю частичную сумму ряда (31)  - через $S_n$. Так как последовательность $\{v_k\}$ монотонна, все разности $v_{k + 1} - v_k$ имеют один знак (все неположительны или все неотрицательны). Но тогда

\begin{equation*}
	S_n = \displaystyle\sum_{k = 1}^n |v_{k + 1} - v_k| = \left| \displaystyle\sum_{k = 1}^n (v_{k + 1} - v_k) \right| = |v_{n + 1} - v_1| \rightarrow |v - v_1| \mbox{ при } n\rightarrow \infty
\end{equation*}

Это  означает, что ряд (31) сходится, его сумма равна $V = |v - v_1|$, а последовательность $\{v_k\}$ имеет ограниченное изменение.

\begin{flushright}
	$\blacksquare$
\end{flushright}

\begin{statm}
	Для того чтобы последовательность имела ограниченное изменение, необходимо и достаточно, чтобы её общий член можно было представить в виде разности двух ограниченных неубывающих последовательностей.
\end{statm}